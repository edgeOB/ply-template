\documentclass[a4paper,11pt]{article}
\usepackage{zh_CN-Adobefonts_external} % Simplified Chinese Support using external fonts (./fonts/zh_CN-Adobe/)
\usepackage{fancyhdr}  % 页眉页脚
\usepackage{minted}    % 代码高亮
\usepackage[colorlinks]{hyperref}  % 目录可跳转
\setlength{\headheight}{15pt}

% 定义页眉页脚
\pagestyle{fancy}
\fancyhf{}
\fancyhead[C]{Algorithm Library by palayutm}
\lfoot{}
\cfoot{\thepage}
\rfoot{}

\author{palayutm}   
\title{Algorithm Library}

\begin{document} 
\maketitle % 封面
\newpage % 换页
\tableofcontents % 目录
\newpage
\section{图论} % 一级标题
\subsection{connected graph} % 二级标题
\subsubsection{割点} % 三级标题
\inputminted[breaklines]{c++}{graph/cutpoint.cc} % 插入代码文件
\subsubsection{割边}
\inputminted[breaklines]{c++}{graph/cutedge.cc}
\subsubsection{强连通分量}
\inputminted[breaklines]{c++}{graph/scc.cc}
\subsubsection{强连通分量缩点}
\inputminted[breaklines]{c++}{graph/scc-ReductionPoints.cc}

\twocolumn  % 分页显示
\newpage
\section{String}
\subsection{KMP}
\inputminted[breaklines]{c++}{string/kmp.cc}

\subsection{Suffix Automaton}
\inputminted[breaklines]{c++}{string/suffix-automaton.cc}

%\newpage
%\section{Others}

\end{document}
